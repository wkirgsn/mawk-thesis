\chapter{Abstract}

%This should include the tasks of your work, your achievements and a few words about the scientific domain where your work is positioned.
The safe operation of permanent magnet synchronous motors is heavily dependent on the heat development inside.
Since these electric motors possess a complex inner structure, it is too cumbersome and cost-intensive to measure important components' temperatures on a sensor basis. 
As production costs and competitive pressure are high - especially in automotive applications - oversizing the implemented material is no option and motor performance is constrained instead, such that operational reliability is not harmed even in a worst case scenario.
In order to enable full utilization of a motor's overload capacities while sustaining its functionality, a real-time temperature estimation with sufficient accuracy is desirable.
In this work, the application of recurrent neural networks featuring memory blocks (LSTM/GRU) are investigated upon their suitability to accurate prediction of important component temperatures inside the aforementioned type of electric motors, which is the first time in literature to our best knowledge.
Having benchmark data available, numerous neural networks were trained and optimized with the aid of the Chainer framework in Python.
A particle swarm optimization was conducted for the models' hyper-parameters on a computing cluster.
It has been found, that performance is comparable to the state-of-the-art estimations of lumped-parameter thermal networks.
Permanent magnet temperatures were predictable with a mean squared error of under \SI{10}{\kelvin\squared}, while accuracy for stator temperatures were even better - the best at under \SI{2}{\kelvin\squared} for the stator yoke.
After determining optimal hyper-parameters for differently targeted models, their individual relevance with respect to model performance has been evaluated by a sensitivity analysis and by surveying the optimization development.
Importance was significantly unequal within a predefined selection of 15 hyper-parameters, so that future research can be focused on relevant aspects in this field.
